\documentclass{article}

\def\Xint#1{\mathchoice
   {\XXint\displaystyle\textstyle{#1}}%
   {\XXint\textstyle\scriptstyle{#1}}%
   {\XXint\scriptstyle\scriptscriptstyle{#1}}%
   {\XXint\scriptscriptstyle\scriptscriptstyle{#1}}%
   \!\int}
\def\XXint#1#2#3{{\setbox0=\hbox{$#1{#2#3}{\int}$}
     \vcenter{\hbox{$#2#3$}}\kern-.5\wd0}}
\def\ddashint{\Xint=}
\def\dashint{\Xint-}

 
\usepackage[margin=1in]{geometry} 
\usepackage{amsmath,amsthm,amssymb}
\usepackage{enumitem}
 \usepackage{graphicx}
 \usepackage{latexsym}
\usepackage{amsfonts}
\usepackage{import, pst-plot}
\newcommand{\N}{\mathbb{N}}
\newcommand{\Z}{\mathbb{Z}}

\newcommand\numberthis{\addtocounter{equation}{1}\tag{\theequation}}

\def\R{\mathbb{R}}
\def\Zp{\mathbb{Z}^+}
 
\begin{document}
 
% --------------------------------------------------------------
%                         Start here
% --------------------------------------------------------------
 
 
%%%%%%%%%%%%%%%%%%%%%%%%%%%%%%%%%
% TITLE PAGE
%%%%%%%%%%%%%%%%%%%%%%%%%%%%%%%%% 
\title{
    \textmd{\Huge{Representations}}\\
    \vspace{0.1in}
    \textbf{Justin Thompson}\\
    \normalsize\vspace{0.1in}\today\\
    \date{}
}


\maketitle

In the first extra credit assignment, we took a matrix given by $$A = \begin{pmatrix} 0 & -1\\1 & 0\\ \end{pmatrix},$$ and found two eigenvalue-eigenvector pairs, 
\begin{align*}
\lambda_1 &= i
&
v_1 &= \begin{pmatrix}1\\-i\end{pmatrix}
&
&\text{and}
&
\lambda_2 &= -i
&
v_2 &= \begin{pmatrix}1\\i\end{pmatrix}.
\end{align*}

The first thing to notice about $A$ is that if $\vec{x}$ is a vector in $\mathbb{R}^2$ (that is, if $\vec{x}$ is a point on the plane), then $A\vec{x}$ is simply $\vec{x}$ rotated 90 degrees counterclockwise about the origin. For instance, suppose that $$\vec{x} = \begin{pmatrix}1\\1\end{pmatrix}.$$ This vector (or point) lives $\sqrt{2}$ units away from the origin, in the first quadrant, on the line $y = x$. If we rotated it 90 degrees counterclockwise about the origin, it would end up at the point $\left(-1, 1\right)$, which is also $\sqrt{2}$ units from the origin, in the second quadrant, on the line $y = -x$. (Draw these points right now!) And it's \textit{really} easy to check that $$\begin{pmatrix}-1\\1\end{pmatrix} = A\begin{pmatrix}1\\1\end{pmatrix}.$$ If you're still not convinced that $A$ rotates each point on the $\mathbb{R}^2$ plane 90 degrees counterclockwise, I encourage you to pick some of your favorite points on the plane, $\vec{x}_1, \vec{x}_2, \dots, \vec{x}_n$, find where a 90 degree rotation \textit{should} take them, and then compute $A\vec{x}_1, A\vec{x}_2, \dots, A\vec{x}_n$. You'll see that $A$ rotates each of your favorite points 90 degrees counterclockwise about the origin. Of course, you could always write a formal proof $\dots$\\

We'll keep all of this in the back of our minds for now. Let's talk about how we represent points in the plane. Thus far, we've thought of each point as a column vector in $\mathbb{R}^2$. Given any point $\left(a, b\right)$ in the plane, we construct a column vector $\vec{v} \in \mathbb{R}^2$, by making the $x$-coordinate the first vector entry and the $y$-coordinate the second entry. That is, $$\left(a, b\right) \rightarrow \begin{pmatrix}a\\b\end{pmatrix}.$$

Let's define a new way to represent a point in the plane. We'll say that if $\left(a, b\right)$ is a point in the plane, then we will represent it like this, $$\left(a, b\right) \rightarrow \begin{pmatrix}a & -b\\b & a\end{pmatrix}.$$ The container for $\left(a, b\right)$ may look different, but the exact same information is encoded in both $$\begin{pmatrix}a\\b\end{pmatrix} \quad \text{and} \quad \begin{pmatrix}a & -b\\b & a\end{pmatrix}.$$ Also, if we're given a matrix of the form $$\begin{pmatrix}x & -y\\y & x\end{pmatrix},$$ we know that this matrix represents the point $\left(x, y\right).$\\

Note that two column vectors cannot be multiplied together to give another column vector, but two $2 \times 2$ matrices can be multiplied to give a $2 \times 2$ matrix. Our new representation gives us a way to multiply two points in the plane. Suppose that $\left(a, b\right)$ and $\left(c, d\right)$ are two points in the plane. Then we can represent these points as the matrices $$ Z_1 = \begin{pmatrix}a & -b\\b & a\end{pmatrix} \quad \text{and} \quad Z_2 = \begin{pmatrix}c & -d\\d & c\end{pmatrix}.$$ Let $Z_3 = Z_1Z_2.$ Then $Z_3$ is given by

\begin{align*}
Z_3 &= Z_1 Z_2\\
&= \begin{pmatrix}a & -b\\b & a\end{pmatrix} \begin{pmatrix}c & -d\\d & c\end{pmatrix}\\
&= \begin{pmatrix}ac - bd & -ad - bc\\bc + ad & -bd + ac\end{pmatrix}\\
&= \begin{pmatrix}ac - bd & -(ad + bc)\\ad + bc & ac - bd\end{pmatrix}\\
&= \begin{pmatrix}x_3 & -y_3\\y_3 & x_3\end{pmatrix}\\
\end{align*}

Since $Z_3$ has the form $\begin{pmatrix}x & -y\\y & x\end{pmatrix}$, then $Z_3$ corresponds to the point $\left(x_3, y_3\right) = \left(ac - bd, ad + bc \right)$. But wait a minute. This looks really familiar. We took the points $\left(a, b\right)$ and $\left(c, d\right)$, multiplied them together and got the point $\left( ac - bd, ad + bc\right)$. That's exactly what we'd get by multiplying the two complex numbers $z_1 = a + ib$ and $z_2 = c + id$ !! Check it out. Let $z_3 = z_1 z_2$. Then,

\begin{align*}
z_3 &= z_1 z_2\\
&= \left(a + ib\right)\left(c + id\right)\\
&= ac + iad + ibc + i^2bd\\
&= ac + iad + ibc - bd\\
&= ac - bd + iad + ibc\\
&= \left(ac - bd\right) + i\left(ad + bc\right).\\
\end{align*}

Since we've defined our representation by $$\left(a, b\right) \rightarrow \begin{pmatrix}a & -b\\b & a\end{pmatrix},$$ and the complex-valued point $z_3$ has $x$-coordinate $\left(ac - bd\right)$ and $y$-coordinate $\left( ad + bc\right)$, we see that $$z_3 = \left(ac - bd, ad+ bc\right) \rightarrow \begin{pmatrix}ac - bd & -\left(ad + bc\right)\\ad + bc & ac - bd\end{pmatrix},$$ so that $z_3$ and $Z_3$ represent the same point. In this way, multiplication of two complex numbers can be represented as the multiplication of two real-valued matrices in $\mathbb{R}^{2 \times 2}$. Let's return to our matrix $A$ from the extra credit assignment.\\

Since $A = \begin{pmatrix}0 & -1 \\ 1 & 0\end{pmatrix}$ has the form $\begin{pmatrix}x & -y \\ y & x\end{pmatrix}$, then $A$ can be represented as a complex number, $A = x + iy$. Substituting the values from the matrix, we find that $A = i.$ And indeed, if you multiply any point $a + ib$ in the plane by $i$, you get a 90 degree rotation counterclockwise about the origin since $$\left(a + ib\right)i = ia + i^2b = -b + ia.$$ 
Pretty cool, huh?





















\end{document}
