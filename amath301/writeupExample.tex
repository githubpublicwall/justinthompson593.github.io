\documentclass{article}

\def\Xint#1{\mathchoice
   {\XXint\displaystyle\textstyle{#1}}%
   {\XXint\textstyle\scriptstyle{#1}}%
   {\XXint\scriptstyle\scriptscriptstyle{#1}}%
   {\XXint\scriptscriptstyle\scriptscriptstyle{#1}}%
   \!\int}
\def\XXint#1#2#3{{\setbox0=\hbox{$#1{#2#3}{\int}$}
     \vcenter{\hbox{$#2#3$}}\kern-.5\wd0}}
\def\ddashint{\Xint=}
\def\dashint{\Xint-}

 
\usepackage[margin=1in]{geometry} 
\usepackage{amsmath,amsthm,amssymb}
\usepackage{enumitem}
 \usepackage{graphicx}
 \usepackage{latexsym}
\usepackage{amsfonts}
\usepackage{import, pst-plot}
\newcommand{\N}{\mathbb{N}}
\newcommand{\Z}{\mathbb{Z}}

\newcommand\numberthis{\addtocounter{equation}{1}\tag{\theequation}}

\def\R{\mathbb{R}}
\def\Zp{\mathbb{Z}^+}
 
\begin{document}
 
% --------------------------------------------------------------
%                         Start here
% --------------------------------------------------------------
 
 
%%%%%%%%%%%%%%%%%%%%%%%%%%%%%%%%%
% TITLE PAGE
%%%%%%%%%%%%%%%%%%%%%%%%%%%%%%%%% 
\title{
    \textmd{\textbf{AMATH 301: Extra Credit 1}}\\
    \vspace{0.1in}
    \textbf{Justin Thompson}\\
    \normalsize\vspace{0.1in}\today\\
    \date{}
}


\maketitle


\textbf{Problem 1}
Let $A \in \mathbb{R}^{2 \times 2}$ be given by $$A = \begin{pmatrix} 
0 & -1\\
1 & 0\\
\end{pmatrix}.$$ We claim that 
$$\vec{v}_1 = \begin{pmatrix} 
1\\
-i\\
\end{pmatrix} \quad \text{and} \quad \vec{v}_2 = \begin{pmatrix} 
1\\
i\\
\end{pmatrix}$$ are eigenvectors of $A$ with corresponding eigenvectors $$\lambda_1 = i \quad \text{and} \quad \lambda_2 = -i.$$
\begin{proof}
Suppose that $A \in \mathbb{R}^{2\times2}$ is given as above. We will first compute the eigenvalues of $A$ by solving the characteristic equation, det$\left( A - \lambda I \right) = 0.$ Observe,
\begin{align*}
\text{det}\left( A - \lambda I \right) &= \text{det}\begin{pmatrix}-\lambda&-1\\1&-\lambda\end{pmatrix}\\
&= \left(-\lambda\right)\left(-\lambda\right) - \left(1\right)\left(-1\right)\\
&= \lambda^2 +1.\\
\end{align*}
By definition of the characteristic equation, we set $\lambda^2 + 1 = 0$ and solve for $\lambda$. This gives us two eigenvalues, $\lambda_1 = i$ and $\lambda_2 = -i$. Now we must find nonzero vectors $\vec{v}_1$ and $\vec{v}_2$ which satisfy
\begin{equation*}
\left( A - \lambda_1 I \right)\vec{v}_1 = \vec{0} \quad \text{and} \quad \left( A - \lambda_2 I \right)\vec{v}_2 = \vec{0}.\\
\end{equation*}
Which is another way of saying
\begin{equation*}
A\vec{v}_1 = \lambda_1 \vec{v}_1 \quad \text{and} \quad A\vec{v}_2 = \lambda_2 \vec{v}_2.\\
\end{equation*}

To solve $\left(A - \lambda_1I\right)\vec{v}_1 = \vec{0}$, we let $\vec{v}_1 = \begin{pmatrix}x_1\\x_2\end{pmatrix}$ so that $\left(A - \lambda_1I\right)\vec{v}_1 = \vec{0}$ becomes
\begin{align*}
\begin{pmatrix}-\lambda_1&-1\\1&-\lambda_1\end{pmatrix}\begin{pmatrix}x_1\\x_2\end{pmatrix} = \begin{pmatrix}0\\0\end{pmatrix}.\\
\end{align*}
This gives us the system of equations
\begin{align*}
-\lambda_1 x_1 - x_2 &= 0\\
x_1 - \lambda_1 x_2 &= 0.\\
\end{align*}
Since we're trying to find a \textit{nonzero} vector which satisfies both of these equations, I'll choose $x_1 = 1$ to make the calculations simple. Substituting $x_1 = 1$ into the first equation in our system gives $$-\lambda_1 - x_2 = 0$$ implying that $$x_2 = -\lambda_1.$$ We have to check that this solution works in the second equation before moving on. Substituting $x_1 = 1$ and $x_2 = -\lambda_1$ into our second equation, $x_1 - \lambda_1 x_2 = 0$, gives 
\begin{align*}
1 - \lambda_1 \left(-\lambda_1\right) = 0\\
1 - i \left(-i\right) = 0\\
1 - \left(-i\right) i = 0\\
1 + i^2 = 0\\
0 = 0\\
\end{align*}
so that the second equation is also satisfied. (Is it true that the second equation will always work out? This would be a good exercise if you're interested!) Since both equations work out, then $$\vec{v}_1 = \begin{pmatrix}1\\-i\end{pmatrix} \quad \text{and} \quad \lambda_1 = i$$ should satisfy $A\vec{v}_1 = \lambda_1 \vec{v}_1$. Let's check.
\begin{align*}
A \vec{v}_1 &= \begin{pmatrix}0&-1\\1&0\end{pmatrix} \begin{pmatrix}1\\-i\end{pmatrix}\\
&= \begin{pmatrix}0 + i\\1 + 0\end{pmatrix}\\
& = \begin{pmatrix}i\\1\end{pmatrix}\\
& = i \begin{pmatrix}1\\-i\end{pmatrix}\\ 
& = \lambda_1 \vec{v}_1.\\ 
\end{align*}
Therefore, we can conclude that $\vec{v}_1$ is an eigenvector of $A$ with eigenvalue $\lambda_1 = i$. Using the exact same method as above, we find that $$\vec{v}_2 = \begin{pmatrix}1\\i \end{pmatrix} \quad \text{and} \quad \lambda_2 = -i$$ are an eigenvector-eigenvalue pair because
\begin{align*}
A \vec{v}_2 &= \begin{pmatrix}0&-1\\1&0\end{pmatrix} \begin{pmatrix}1\\i\end{pmatrix}\\
&= \begin{pmatrix}0 - i\\1 + 0\end{pmatrix}\\
&= \begin{pmatrix}-i\\1\end{pmatrix}\\
& = -i \begin{pmatrix}1\\i\end{pmatrix}\\ 
& = \lambda_2 \vec{v}_2.\\ 
\end{align*}
This is what we wanted to show.
\end{proof}



\end{document}


























